\documentclass[12pt]{article}
\usepackage{graphicx,amsmath,amsfonts,amssymb,epsfig,euscript,enumerate}
\usepackage[T1]{fontenc}
\usepackage[utf8x]{inputenc}

\newtheorem{ejer}{Ejercicio}
\newcommand{\bej}{\vspace{1pt}\begin{ejer}\rm}
\newcommand{\fej}{\end{ejer}}

\newcommand{\R}{\mathbb{R}}
\newcommand{\C}{\mathbb{C}}
\def\dt{\Delta t}
\def\dx{\Delta x}

\topmargin-1cm \vsize 29.5cm \hsize 21cm
\setlength{\textwidth}{16.50cm}\setlength{\textheight}{23cm}
\setlength{\oddsidemargin}{0.0cm}
\setlength{\evensidemargin}{0.0cm}

\begin{document}
\centerline{{\small Universidad de Buenos Aires - Facultad de Ciencias Exactas y Naturales - Depto. de Matemática}}
 
 \vskip 0.2cm
 \hrulefill
 \vskip 0.2cm

 \centerline{{\bf\Huge {\sc Análisis Numérico}}}
 \vskip 0.2cm
 \centerline{\ttfamily Segundo Cuatrimestre 2020}
 \hrulefill

 \bigskip
 \centerline{\bf Trabajo práctico grupal: Diferencias Finitas en 2D}
 \bigskip
 
La modalidad de este trabajo práctico grupal será la de contribuir, cada uno, a un proyecto de código abierto que consiste en una suite de python para la resolución de problemas de Ecuaciones en Derivadas Parciales en 2D. La modalidad de aprobación será lograr haber contribuido código no-trivial para la resolución de al menos un problema.

El desarrollo del código se organizará sobre la base de la resolución de problemas de una complejidad creciente, y donde pondremos énfasis en la claridad y modularidad del código, de forma que resulte reutilizable para la resolución de otros problemas. 

Para estas primeros problemas trabajaremos en un rectángulo de referencia. Para cada función de dos variables $v_{ij}=v(x_i,y_j)$ definimos el vector $\hat{v}$ como el desarrollo por columnas de la matriz $v_{ij}$.

Problemas a implementar. 
\begin{enumerate}[(a)]
 \item Modifique el método explícito para la ecuación del calor para resolver el caso de las condiciones de borde periódicas.
 
 \item idem con el método implícito.
 
 \item Implementar un método explícito para simular un sistema no-lineal de reacción-difusión, de la forma
 \begin{equation*}
  \begin{array}{cc}
   U_t = & \mu_u \Delta U - UV^2 + F(1-U) \\
   V_t = & \mu_v \Delta V + UV^2 - (F+k)V\\
  \end{array}
 \end{equation*}
 con condiciones de borde periódicas, y donde $F,k$ son parámetros dados. Experimente si, con condiciones iniciales aleatorias, el sistema llega a un estado estacionario. 
 
 \item Implemente un método implícito en la difusión y explícito en los términos no-lineales para resolver el problema anterior.
 
 \item Implemente un método totalmente implícito de orden 2 en el tiempo para la ecuación del calor basado en la fórmula BDF2.
 
 \item Implemente un método implícito-explícito de orden 2 basandose en el caso anterior y una fórmula de extrapolación de orden 2 (consultar).
\end{enumerate}


\end{document}
